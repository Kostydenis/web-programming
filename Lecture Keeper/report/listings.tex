\captionsetup[figure]{labelformat=mylisting}

\chapter{Листинги приложения}

Все исходные коды приложения можно разделить на серверную и клиентскую части. В серверную часть входят файлы, запускаемые на сервере и учавствующие в процессе разработки, а в клиентскую - то, что отображается пользователю - шаблоны и стили.

\noindent
\ttfamily
\hlstd{}\hllin{\ \ \ 1\ }\hlkwa{from\ }\hlstd{flask\ }\hlkwa{import\ }\hlstd{Flask\\
\hllin{\ \ \ 2\ }\\
\hllin{\ \ \ 3\ }app\ }\hlopt{=\ }\hlstd{}\hlkwd{Flask}\hlstd{}\hlopt{(}\hlstd{\textunderscore \textunderscore name\textunderscore \textunderscore }\hlopt{)}\\
\hllin{\ \ \ 4\ }\hlstd{app}\hlopt{.}\hlstd{config}\hlopt{.}\hlstd{}\hlkwd{from\textunderscore object}\hlstd{}\hlopt{(}\hlstd{}\hlstr{'config'}\hlstd{}\hlopt{)}\\
\hllin{\ \ \ 5\ }\hlstd{}\hlkwa{from\ }\hlstd{app\ }\hlkwa{import\ }\hlstd{views}\\
\mbox{}
\normalfont
\normalsize

\begin{figure}[h!]
	\caption{\_\_init\_\_.py}
\end{figure}

\noindent
\ttfamily
\hlstd{}\hllin{\ \ \ 1\ }\\
\hllin{\ \ \ 2\ }\hlkwa{from\ }\hlstd{flask\ }\hlkwa{import\ }\hlstd{render\textunderscore template}\\
\hllin{\ \ \ 3\ }\hlkwa{from\ }\hlstd{flask\ }\hlkwa{import\ }\hlstd{send\textunderscore file}\\
\hllin{\ \ \ 4\ }\hlkwa{from\ }\hlstd{app\ }\hlkwa{import\ }\hlstd{app}\\
\hllin{\ \ \ 5\ }\hlslc{\#\ import\ disciplines}\\
\hllin{\ \ \ 6\ }\hlstd{}\\
\hllin{\ \ \ 7\ }\hlkwa{class\ }\hlstd{Discipline}\hlopt{:}\\
\hllin{\ \ \ 8\ }\hlstd{}\hlstd{\ \ \ \ }\hlstd{}\hlkwa{def\ }\hlstd{}\hlkwd{\textunderscore \textunderscore init\textunderscore \textunderscore }\hlstd{}\hlopt{(}\hlstd{self}\hlopt{,\ }\hlstd{code}\hlopt{,\ }\hlstd{name}\hlopt{,\ }\hlstd{pdf}\hlopt{):}\\
\hllin{\ \ \ 9\ }\hlstd{}\hlstd{\ \ \ \ \ \ \ \ }\hlstd{self}\hlopt{.}\hlstd{code\ }\hlopt{=\ }\hlstd{code\\
\hllin{\ \ 10\ }}\hlstd{\ \ \ \ \ \ \ \ }\hlstd{self}\hlopt{.}\hlstd{name\ }\hlopt{=\ }\hlstd{name\\
\hllin{\ \ 11\ }}\hlstd{\ \ \ \ \ \ \ \ }\hlstd{self}\hlopt{.}\hlstd{pdf\ }\hlopt{=\ }\hlstd{pdf\\
\hllin{\ \ 12\ }\\
\hllin{\ \ 13\ }web\ }\hlopt{=\ }\hlstd{}\hlkwd{Discipline}\hlstd{}\hlopt{(}\hlstd{}\hlstr{'web'}\hlstd{}\hlopt{,\ }\hlstd{}\hlstr{'Веб{-}программирование'}\hlstd{}\hlopt{,\ }\hlstd{}\hlstr{'webPDF'}\hlstd{}\hlopt{)}\\
\hllin{\ \ 14\ }\hlstd{ics\ }\hlopt{=\ }\hlstd{}\hlkwd{Discipline}\hlstd{}\hlopt{(}\hlstd{}\hlstr{'ics'}\hlstd{}\hlopt{,\ }\hlstd{}\hlstr{'Проектирование\ АСОИУ'}\hlstd{}\hlopt{,\ }\hlstd{}\hlstr{'icsPDF'}\hlstd{}\hlopt{)}\\
\hllin{\ \ 15\ }\hlstd{electr\ }\hlopt{=\ }\hlstd{}\hlkwd{Discipline}\hlstd{}\hlopt{(}\hlstd{}\hlstr{'electr'}\hlstd{}\hlopt{,\ }\hlstd{}\hlstr{'Схемотехника'}\hlstd{}\hlopt{,\ }\hlstd{}\hlstr{'electrPDF'}\hlstd{}\hlopt{)}\\
\hllin{\ \ 16\ }\hlstd{\\
\hllin{\ \ 17\ }discipline\textunderscore list\ }\hlopt{=\ {[}}\hlstd{web}\hlopt{,\ }\hlstd{ics}\hlopt{,\ }\hlstd{electr}\hlopt{{]}}\\
\hllin{\ \ 18\ }\hlstd{}\\
\hllin{\ \ 19\ }\hlkwb{@app}\hlstd{}\hlopt{.}\hlstd{}\hlkwd{route}\hlstd{}\hlopt{(}\hlstd{}\hlstr{'/'}\hlstd{}\hlopt{)}\\
\hllin{\ \ 20\ }\hlstd{}\hlkwb{@app}\hlstd{}\hlopt{.}\hlstd{}\hlkwd{route}\hlstd{}\hlopt{(}\hlstd{}\hlstr{'/index'}\hlstd{}\hlopt{)}\\
\hllin{\ \ 21\ }\hlstd{}\hlkwa{def\ }\hlstd{}\hlkwd{index}\hlstd{}\hlopt{():}\\
\hllin{\ \ 22\ }\hlstd{}\hlstd{\ \ \ \ }\hlstd{}\hlkwa{return\ }\hlstd{}\hlkwd{render\textunderscore template}\hlstd{}\hlopt{(}\hlstd{}\hlstr{'index.html'}\hlstd{}\hlopt{,\ }\hlstd{discipline\textunderscore list\ }\hlopt{=\ }\hlstd{discipline\textunderscore list}\hlopt{)}\\
\hllin{\ \ 23\ }\hlstd{}\\
\hllin{\ \ 24\ }\hlkwb{@app}\hlstd{}\hlopt{.}\hlstd{}\hlkwd{route}\hlstd{}\hlopt{(}\hlstd{}\hlstr{'/about'}\hlstd{}\hlopt{)}\\
\hllin{\ \ 25\ }\hlstd{}\hlkwa{def\ }\hlstd{}\hlkwd{about}\hlstd{}\hlopt{():}\\
\hllin{\ \ 26\ }\hlstd{\ }\hlkwa{return\ }\hlstd{}\hlkwd{render\textunderscore template}\hlstd{}\hlopt{(}\hlstd{}\hlstr{'about.html'}\hlstd{}\hlopt{,\ }\hlstd{discipline\textunderscore list\ }\hlopt{=\ }\hlstd{discipline\textunderscore list}\hlopt{)}\\
\hllin{\ \ 27\ }\hlstd{}\\
\hllin{\ \ 28\ }\hlkwb{@app}\hlstd{}\hlopt{.}\hlstd{}\hlkwd{route}\hlstd{}\hlopt{(}\hlstd{}\hlstr{'/web'}\hlstd{}\hlopt{)}\\
\hllin{\ \ 29\ }\hlstd{}\hlkwa{def\ }\hlstd{}\hlkwd{web}\hlstd{}\hlopt{():}\\
\hllin{\ \ 30\ }\hlstd{\ }\hlkwa{return\ }\hlstd{}\hlkwd{render\textunderscore template}\hlstd{}\hlopt{(}\hlstd{}\hlstr{'/web/index.html'}\hlstd{}\hlopt{,\ }\hlstd{discipline\textunderscore list\ }\hlopt{=\ }\hlstd{discipline\textunderscore list}\hlopt{)}\\
\hllin{\ \ 31\ }\hlstd{}\\
\hllin{\ \ 32\ }\hlkwb{@app}\hlstd{}\hlopt{.}\hlstd{}\hlkwd{route}\hlstd{}\hlopt{(}\hlstd{}\hlstr{'/webPDF'}\hlstd{}\hlopt{)}\\
\hllin{\ \ 33\ }\hlstd{}\hlkwa{def\ }\hlstd{}\hlkwd{webPDF}\hlstd{}\hlopt{():}\\
\hllin{\ \ 34\ }\hlstd{\ }\hlkwa{return\ }\hlstd{}\hlkwd{send\textunderscore file}\hlstd{}\hlopt{(}\hlstd{}\hlstr{'static/pdfs/web.pdf'}\hlstd{}\hlopt{,\ }\hlstd{as\textunderscore attachment}\hlopt{=}\hlstd{}\hlkwa{True}\hlstd{}\hlopt{)}\\
\hllin{\ \ 35\ }\hlstd{}\\
\hllin{\ \ 36\ }\hlkwb{@app}\hlstd{}\hlopt{.}\hlstd{}\hlkwd{route}\hlstd{}\hlopt{(}\hlstd{}\hlstr{'/ics'}\hlstd{}\hlopt{)}\\
\hllin{\ \ 37\ }\hlstd{}\hlkwa{def\ }\hlstd{}\hlkwd{ics}\hlstd{}\hlopt{():}\\
\hllin{\ \ 38\ }\hlstd{\ }\hlkwa{return\ }\hlstd{}\hlkwd{render\textunderscore template}\hlstd{}\hlopt{(}\hlstd{}\hlstr{'/ics/index.html'}\hlstd{}\hlopt{,\ }\hlstd{discipline\textunderscore list\ }\hlopt{=\ }\hlstd{discipline\textunderscore list}\hlopt{)}\\
\hllin{\ \ 39\ }\hlstd{}\\
\hllin{\ \ 40\ }\hlkwb{@app}\hlstd{}\hlopt{.}\hlstd{}\hlkwd{route}\hlstd{}\hlopt{(}\hlstd{}\hlstr{'/icsPDF'}\hlstd{}\hlopt{)}\\
\hllin{\ \ 41\ }\hlstd{}\hlkwa{def\ }\hlstd{}\hlkwd{icsPDF}\hlstd{}\hlopt{():}\\
\hllin{\ \ 42\ }\hlstd{\ }\hlkwa{return\ }\hlstd{}\hlkwd{send\textunderscore file}\hlstd{}\hlopt{(}\hlstd{}\hlstr{'static/pdfs/ics.pdf'}\hlstd{}\hlopt{,\ }\hlstd{as\textunderscore attachment}\hlopt{=}\hlstd{}\hlkwa{True}\hlstd{}\hlopt{)}\\
\hllin{\ \ 43\ }\hlstd{}\\
\hllin{\ \ 44\ }\hlkwb{@app}\hlstd{}\hlopt{.}\hlstd{}\hlkwd{route}\hlstd{}\hlopt{(}\hlstd{}\hlstr{'/electr'}\hlstd{}\hlopt{)}\\
\hllin{\ \ 45\ }\hlstd{}\hlkwa{def\ }\hlstd{}\hlkwd{electr}\hlstd{}\hlopt{():}\\
\hllin{\ \ 46\ }\hlstd{\ }\hlkwa{return\ }\hlstd{}\hlkwd{render\textunderscore template}\hlstd{}\hlopt{(}\hlstd{}\hlstr{'electr/index.html'}\hlstd{}\hlopt{,\ }\hlstd{discipline\textunderscore list\ }\hlopt{=\ }\hlstd{discipline\textunderscore list}\hlopt{)}\\
\hllin{\ \ 47\ }\hlstd{}\\
\hllin{\ \ 48\ }\hlkwb{@app}\hlstd{}\hlopt{.}\hlstd{}\hlkwd{route}\hlstd{}\hlopt{(}\hlstd{}\hlstr{'/electrPDF'}\hlstd{}\hlopt{)}\\
\hllin{\ \ 49\ }\hlstd{}\hlkwa{def\ }\hlstd{}\hlkwd{electrPDF}\hlstd{}\hlopt{():}\\
\hllin{\ \ 50\ }\hlstd{\ }\hlkwa{return\ }\hlstd{}\hlkwd{send\textunderscore file}\hlstd{}\hlopt{(}\hlstd{}\hlstr{'static/pdfs/electr.pdf'}\hlstd{}\hlopt{,\ }\hlstd{as\textunderscore attachment}\hlopt{=}\hlstd{}\hlkwa{True}\hlstd{}\hlopt{)}\hlstd{}\\
\mbox{}
\normalfont
\normalsize

\begin{figure}[h!]
	\caption{views.py}
\end{figure}

	\noindent
\ttfamily
\hlstd{}\hllin{\ \ \ 1\ }\hlkwa{from\ }\hlstd{app\ }\hlkwa{import\ }\hlstd{app}\\
\hllin{\ \ \ 2\ }\\
\hllin{\ \ \ 3\ }\hlslc{\#\ app.run(host='192.168.0.199')}\\
\hllin{\ \ \ 4\ }\hlstd{}\hlslc{\#\ app.run(debug=True)}\\
\hllin{\ \ \ 5\ }\hlstd{\\
\hllin{\ \ \ 6\ }app}\hlopt{.}\hlstd{}\hlkwd{run}\hlstd{}\hlopt{()}\hlstd{}\\
\mbox{}
\normalfont
\normalsize

\begin{figure}[h!]
	\caption{run.py}
\end{figure}

\section{Шаблоны}

%\noindent
\ttfamily
\hlstd{}\hllin{\ \ \ 1\ }\hlkwa{\usebox{\hlboxlessthan}!DOCTYPE\ }\hlstd{html}\hlkwa{\usebox{\hlboxgreaterthan}}\\
\hllin{\ \ \ 2\ }\hlstd{}\hlkwa{\usebox{\hlboxlessthan}head\usebox{\hlboxgreaterthan}}\\
\hllin{\ \ \ 3\ }\hlstd{}\hlstd{\ \ \ \ }\hlstd{}\hlkwa{\usebox{\hlboxlessthan}title\usebox{\hlboxgreaterthan}}\hlstd{Lecture\ Keeper}\hlkwa{\usebox{\hlboxlessthan}/title\usebox{\hlboxgreaterthan}}\\
\hllin{\ \ \ 4\ }\hlstd{}\hlstd{\ \ \ \ }\hlstd{}\hlkwa{\usebox{\hlboxlessthan}meta\ }\hlstd{}\hlkwb{charset}\hlstd{=}\hlstr{"UTF{-}8"}\hlstd{\ }\hlkwa{/\usebox{\hlboxgreaterthan}}\\
\hllin{\ \ \ 5\ }\hlstd{}\hlstd{\ \ \ \ }\hlstd{}\hlkwa{\usebox{\hlboxlessthan}meta\ }\hlstd{}\hlkwb{name}\hlstd{=}\hlstr{"viewport"}\hlstd{\ }\hlkwb{content}\hlstd{=}\hlstr{"width=device{-}width,\ initial{-}scale=1.0"}\hlstd{\ }\hlkwa{/\usebox{\hlboxgreaterthan}}\\
\hllin{\ \ \ 6\ }\hlstd{}\hlstd{\ \ \ \ }\hlstd{}\hlkwa{\usebox{\hlboxlessthan}link\ }\hlstd{}\hlkwb{rel}\hlstd{=}\hlstr{"stylesheet"}\hlstd{\ }\hlkwb{type}\hlstd{=}\hlstr{"text/css"}\hlstd{\ }\hlkwb{href}\hlstd{=}\hlstr{"\usebox{\hlboxopenbrace}\usebox{\hlboxopenbrace}\ url\textunderscore for('static',\ filename='\textunderscore css/main.css')\ \usebox{\hlboxclosebrace}\usebox{\hlboxclosebrace}"}\hlstd{\ }\hlkwa{/\usebox{\hlboxgreaterthan}}\\
\hllin{\ \ \ 7\ }\hlstd{}\hlstd{\ \ \ \ }\hlstd{}\hlkwa{\usebox{\hlboxlessthan}script\ }\hlstd{}\hlkwb{type}\hlstd{=}\hlstr{"text/x{-}mathjax{-}config"}\hlstd{}\hlkwa{\usebox{\hlboxgreaterthan}}\hlstd{MathJax.Hub.Config(\usebox{\hlboxopenbrace}tex2jax:\usebox{\hlboxopenbrace}inlineMath:{[}{[}'\$\$\$','\$\$\$'{]}{]}\usebox{\hlboxclosebrace}\usebox{\hlboxclosebrace});}\hlkwa{\usebox{\hlboxlessthan}/script\usebox{\hlboxgreaterthan}}\\
\hllin{\ \ \ 8\ }\hlstd{}\hlstd{\ \ \ \ }\hlstd{}\hlkwa{\usebox{\hlboxlessthan}script\ }\hlstd{}\hlkwb{src}\hlstd{=}\hlstr{"http://cdn.mathjax.org/mathjax/latest/MathJax.js?config=TeX{-}AMS{-}MML\textunderscore HTMLorMML"}\hlstd{}\hlkwa{\usebox{\hlboxgreaterthan}\usebox{\hlboxlessthan}/script\usebox{\hlboxgreaterthan}}\\
\hllin{\ \ \ 9\ }\hlstd{}\hlkwa{\usebox{\hlboxlessthan}/head\usebox{\hlboxgreaterthan}}\\
\hllin{\ \ 10\ }\hlstd{}\hlkwa{\usebox{\hlboxlessthan}body\usebox{\hlboxgreaterthan}}\\
\hllin{\ \ 11\ }\hlstd{}\hlstd{\ \ \ \ }\hlstd{}\hlkwa{\usebox{\hlboxlessthan}div\ }\hlstd{}\hlkwb{class}\hlstd{=}\hlstr{"hero"}\hlstd{}\hlkwa{\usebox{\hlboxgreaterthan}\usebox{\hlboxlessthan}/div\usebox{\hlboxgreaterthan}}\\
\hllin{\ \ 12\ }\hlstd{}\hlstd{\ \ \ \ }\hlstd{}\hlkwa{\usebox{\hlboxlessthan}div\ }\hlstd{}\hlkwb{class}\hlstd{=}\hlstr{"app\textunderscore bar"}\hlstd{}\hlkwa{\usebox{\hlboxgreaterthan}}\\
\hllin{\ \ 13\ }\hlstd{}\hlstd{\ \ \ \ \ \ \ \ }\hlstd{}\hlkwa{\usebox{\hlboxlessthan}ul\ }\hlstd{}\hlkwb{class}\hlstd{=}\hlstr{"pull{-}left"}\hlstd{}\hlkwa{\usebox{\hlboxgreaterthan}}\\
\hllin{\ \ 14\ }\hlstd{}\hlstd{\ \ \ \ \ \ \ \ \ \ \ \ }\hlstd{}\hlkwa{\usebox{\hlboxlessthan}li\ }\hlstd{}\hlkwb{class}\hlstd{=}\hlstr{"fa\ fa{-}bars\ menu\textunderscore btn"}\hlstd{\ }\hlkwb{onclick}\hlstd{=}\hlstr{"show\textunderscore menu()"}\hlstd{}\hlkwa{\usebox{\hlboxgreaterthan}\usebox{\hlboxlessthan}/li\usebox{\hlboxgreaterthan}}\\
\hllin{\ \ 15\ }\hlstd{}\hlstd{\ \ \ \ \ \ \ \ \ \ \ \ }\hlstd{}\hlkwa{\usebox{\hlboxlessthan}li\usebox{\hlboxgreaterthan}\usebox{\hlboxlessthan}a\ }\hlstd{}\hlkwb{href}\hlstd{=}\hlstr{"/"}\hlstd{}\hlkwa{\usebox{\hlboxgreaterthan}}\hlstd{Home}\hlkwa{\usebox{\hlboxlessthan}/a\usebox{\hlboxgreaterthan}\usebox{\hlboxlessthan}/li\usebox{\hlboxgreaterthan}}\\
\hllin{\ \ 16\ }\hlstd{}\hlstd{\ \ \ \ \ \ \ \ }\hlstd{}\hlkwa{\usebox{\hlboxlessthan}/ul\usebox{\hlboxgreaterthan}}\\
\hllin{\ \ 17\ }\hlstd{}\hlstd{\ \ \ \ \ \ \ \ }\hlstd{}\hlkwa{\usebox{\hlboxlessthan}ul\ }\hlstd{}\hlkwb{class}\hlstd{=}\hlstr{"pull{-}right"}\hlstd{}\hlkwa{\usebox{\hlboxgreaterthan}}\\
\hllin{\ \ 18\ }\hlstd{}\hlstd{\ \ \ \ \ \ \ \ \ \ \ \ }\hlstd{}\hlkwa{\usebox{\hlboxlessthan}li\usebox{\hlboxgreaterthan}\usebox{\hlboxlessthan}a\ }\hlstd{}\hlkwb{href}\hlstd{=}\hlstr{"/about"}\hlstd{}\hlkwa{\usebox{\hlboxgreaterthan}}\hlstd{About}\hlkwa{\usebox{\hlboxlessthan}/a\usebox{\hlboxgreaterthan}\usebox{\hlboxlessthan}/li\usebox{\hlboxgreaterthan}}\\
\hllin{\ \ 19\ }\hlstd{}\hlstd{\ \ \ \ \ \ \ \ \ \ \ \ }\hlstd{}\hlkwa{\usebox{\hlboxlessthan}li\ }\hlstd{}\hlkwb{onclick}\hlstd{=}\hlstr{"location.href='\#'"}\hlstd{}\hlkwa{\usebox{\hlboxgreaterthan}}\\
\hllin{\ \ 20\ }\hlstd{}\hlstd{\ \ \ \ \ \ \ \ \ \ \ \ \ \ \ \ }\hlstd{}\hlkwa{\usebox{\hlboxlessthan}svg\ }\hlstd{}\hlkwb{viewBox}\hlstd{=}\hlstr{"0\ 0\ 24\ 24"}\hlstd{}\hlkwa{\usebox{\hlboxgreaterthan}}\\
\hllin{\ \ 21\ }\hlstd{}\hlstd{\ \ \ \ \ \ \ \ \ \ \ \ \ \ \ \ \ \ \ \ }\hlstd{}\hlkwa{\usebox{\hlboxlessthan}path\ }\hlstd{}\hlkwb{fill}\hlstd{=}\hlstr{"\#ffffff"}\hlstd{\ }\hlkwb{d}\hlstd{=}\hlstr{"M13,20H11V8L5.5,13.5L4.08,12.08L12,4.16L19.92,12.08L18.5,13.5L13,8V20Z"}\hlstd{\ }\hlkwa{/\usebox{\hlboxgreaterthan}}\\
\hllin{\ \ 22\ }\hlstd{}\hlstd{\ \ \ \ \ \ \ \ \ \ \ \ \ \ \ \ }\hlstd{}\hlkwa{\usebox{\hlboxlessthan}/svg\usebox{\hlboxgreaterthan}}\\
\hllin{\ \ 23\ }\hlstd{}\hlstd{\ \ \ \ \ \ \ \ \ \ \ \ }\hlstd{}\hlkwa{\usebox{\hlboxlessthan}/li\usebox{\hlboxgreaterthan}}\\
\hllin{\ \ 24\ }\hlstd{}\hlstd{\ \ \ \ \ \ \ \ }\hlstd{}\hlkwa{\usebox{\hlboxlessthan}/ul\usebox{\hlboxgreaterthan}}\\
\hllin{\ \ 25\ }\hlstd{}\hlstd{\ \ \ \ }\hlstd{}\hlkwa{\usebox{\hlboxlessthan}/div\usebox{\hlboxgreaterthan}}\\
\hllin{\ \ 26\ }\hlstd{\\
\hllin{\ \ 27\ }}\hlstd{\ \ \ \ }\hlstd{}\hlkwa{\usebox{\hlboxlessthan}div\ }\hlstd{}\hlkwb{class}\hlstd{=}\hlstr{"darken\ darken\textunderscore hide"}\hlstd{\ }\hlkwb{onclick}\hlstd{=}\hlstr{"hide\textunderscore menu()"}\hlstd{\ }\hlkwb{id}\hlstd{=}\hlstr{"darken"}\hlstd{}\hlkwa{\usebox{\hlboxgreaterthan}\usebox{\hlboxlessthan}/div\usebox{\hlboxgreaterthan}}\\
\hllin{\ \ 28\ }\hlstd{\\
\hllin{\ \ 29\ }}\hlstd{\ \ \ \ }\hlstd{}\hlkwa{\usebox{\hlboxlessthan}div\ }\hlstd{}\hlkwb{class}\hlstd{=}\hlstr{"menu\ menu\textunderscore hide"}\hlstd{\ }\hlkwb{id}\hlstd{=}\hlstr{"menu"}\hlstd{}\hlkwa{\usebox{\hlboxgreaterthan}}\\
\hllin{\ \ 30\ }\hlstd{}\hlstd{\ \ \ \ \ \ \ \ }\hlstd{}\hlkwa{\usebox{\hlboxlessthan}h1\usebox{\hlboxgreaterthan}\usebox{\hlboxlessthan}a\ }\hlstd{}\hlkwb{href}\hlstd{=}\hlstr{"/"}\hlstd{}\hlkwa{\usebox{\hlboxgreaterthan}}\hlstd{Lecture\ Keeper}\hlkwa{\usebox{\hlboxlessthan}/a\usebox{\hlboxgreaterthan}\usebox{\hlboxlessthan}/h1\usebox{\hlboxgreaterthan}}\\
\hllin{\ \ 31\ }\hlstd{}\hlstd{\ \ \ \ \ \ \ \ }\hlstd{}\hlkwa{\usebox{\hlboxlessthan}div\ }\hlstd{}\hlkwb{class}\hlstd{=}\hlstr{"divider"}\hlstd{}\hlkwa{\usebox{\hlboxgreaterthan}\usebox{\hlboxlessthan}/div\usebox{\hlboxgreaterthan}}\\
\hllin{\ \ 32\ }\hlstd{}\hlstd{\ \ \ \ \ \ \ \ }\hlstd{}\hlkwa{\usebox{\hlboxlessthan}ul\usebox{\hlboxgreaterthan}}\\
\hllin{\ \ 33\ }\hlstd{}\hlstd{\ \ \ \ \ \ \ \ \ \ \ \ }\hlstd{\usebox{\hlboxopenbrace}\%\ for\ discipline\ in\ discipline\textunderscore list\ \%\usebox{\hlboxclosebrace}\\
\hllin{\ \ 34\ }}\hlstd{\ \ \ \ \ \ \ \ \ \ \ \ }\hlstd{}\hlkwa{\usebox{\hlboxlessthan}li\usebox{\hlboxgreaterthan}\usebox{\hlboxlessthan}a\ }\hlstd{}\hlkwb{href}\hlstd{=}\hlstr{"\usebox{\hlboxopenbrace}\usebox{\hlboxopenbrace}\ discipline.code\ \usebox{\hlboxclosebrace}\usebox{\hlboxclosebrace}"}\hlstd{}\hlkwa{\usebox{\hlboxgreaterthan}}\hlstd{\usebox{\hlboxopenbrace}\usebox{\hlboxopenbrace}\ discipline.name\ \usebox{\hlboxclosebrace}\usebox{\hlboxclosebrace}}\hlkwa{\usebox{\hlboxlessthan}/a\usebox{\hlboxgreaterthan}\usebox{\hlboxlessthan}/li\usebox{\hlboxgreaterthan}}\\
\hllin{\ \ 35\ }\hlstd{}\hlstd{\ \ \ \ \ \ \ \ \ \ \ \ }\hlstd{\usebox{\hlboxopenbrace}\%\ endfor\ \%\usebox{\hlboxclosebrace}\\
\hllin{\ \ 36\ }}\hlstd{\ \ \ \ \ \ \ \ }\hlstd{}\hlkwa{\usebox{\hlboxlessthan}/ul\usebox{\hlboxgreaterthan}}\\
\hllin{\ \ 37\ }\hlstd{}\hlstd{\ \ \ \ }\hlstd{}\hlkwa{\usebox{\hlboxlessthan}/div\usebox{\hlboxgreaterthan}}\\
\hllin{\ \ 38\ }\hlstd{\\
\hllin{\ \ 39\ }}\hlstd{\ \ \ \ }\hlstd{}\hlkwa{\usebox{\hlboxlessthan}div\ }\hlstd{}\hlkwb{class}\hlstd{=}\hlstr{"content"}\hlstd{}\hlkwa{\usebox{\hlboxgreaterthan}}\\
\hllin{\ \ 40\ }\hlstd{}\hlstd{\ \ \ \ \ \ \ \ }\hlstd{}\hlkwa{\usebox{\hlboxlessthan}a\ }\hlstd{}\hlkwb{href}\hlstd{=}\hlstr{"https://github.com/Kostydenis/web{-}programming/tree/master/Lecture\%20Keeper"}\hlstd{}\hlkwa{\usebox{\hlboxgreaterthan}}\\
\hllin{\ \ 41\ }\hlstd{}\hlstd{\ \ \ \ \ \ \ \ \ \ \ \ }\hlstd{}\hlkwa{\usebox{\hlboxlessthan}img\ }\hlstd{}\hlkwb{style}\hlstd{=}\hlstr{"position:\ absolute;\ top:\ 0;\ right:\ 0;\ border:\ 0;"}\hlstd{\ }\hlkwb{src}\hlstd{=}\hlstr{"https://camo.githubusercontent.com/e7bbb0521b397edbd5fe43e7f760759336b5e05f/68747470733a2f2f73332e616d617a6f6e6177732e636f6d2f6769746875622f726962626f6e732f666f726b6d655f72696768745f677265656e5f3030373230302e706e67"}\hlstd{\ }\hlkwb{alt}\hlstd{=}\hlstr{"Fork\ me\ on\ GitHub"}\hlstd{\ }\hlkwb{data{-}canonical{-}src}\hlstd{=}\hlstr{"https://s3.amazonaws.com/github/ribbons/forkme\textunderscore right\textunderscore green\textunderscore 007200.png"}\hlstd{}\hlkwa{\usebox{\hlboxgreaterthan}}\\
\hllin{\ \ 42\ }\hlstd{}\hlstd{\ \ \ \ \ \ \ \ }\hlstd{}\hlkwa{\usebox{\hlboxlessthan}/a\usebox{\hlboxgreaterthan}}\\
\hllin{\ \ 43\ }\hlstd{}\hlstd{\ \ \ \ }\hlstd{\\
\hllin{\ \ 44\ }}\hlstd{\ \ \ \ \ \ \ \ }\hlstd{}\hlkwa{\usebox{\hlboxlessthan}h1\ }\hlstd{}\hlkwb{class}\hlstd{=}\hlstr{"non\textunderscore lecture"}\hlstd{}\hlkwa{\usebox{\hlboxgreaterthan}}\hlstd{Lecture\ Keeper}\hlkwa{\usebox{\hlboxlessthan}/h1\usebox{\hlboxgreaterthan}}\\
\hllin{\ \ 45\ }\hlstd{}\hlstd{\ \ \ \ \ \ \ \ }\hlstd{}\hlkwa{\usebox{\hlboxlessthan}div\ }\hlstd{}\hlkwb{class}\hlstd{=}\hlstr{"divider"}\hlstd{}\hlkwa{\usebox{\hlboxgreaterthan}\usebox{\hlboxlessthan}/div\usebox{\hlboxgreaterthan}}\\
\hllin{\ \ 46\ }\hlstd{}\hlstd{\ \ \ \ \ \ \ \ }\hlstd{}\hlcom{\usebox{\hlboxlessthan}!{-}{-}\ \usebox{\hlboxlessthan}div\ class="non\textunderscore lecture\textunderscore divider"\usebox{\hlboxgreaterthan}\usebox{\hlboxlessthan}/div\usebox{\hlboxgreaterthan}\ {-}{-}\usebox{\hlboxgreaterthan}}\hlstd{\\
\hllin{\ \ 47\ }}\hlstd{\ \ \ \ \ \ \ \ }\hlstd{}\hlkwa{\usebox{\hlboxlessthan}h2\ }\hlstd{}\hlkwb{class}\hlstd{=}\hlstr{"non\textunderscore lecture"}\hlstd{}\hlkwa{\usebox{\hlboxgreaterthan}}\hlstd{Лекции\ и\ другие\ материалы\ АСУб{-}}\hlnum{12}\hlstd{{-}}\hlnum{1}\hlstd{}\hlkwa{\usebox{\hlboxlessthan}/h2\usebox{\hlboxgreaterthan}}\\
\hllin{\ \ 48\ }\hlstd{\\
\hllin{\ \ 49\ }}\hlstd{\ \ \ \ \ \ \ \ }\hlstd{\usebox{\hlboxopenbrace}\%\ block\ disciplines\ \%\usebox{\hlboxclosebrace}\usebox{\hlboxopenbrace}\%\ endblock\ \%\usebox{\hlboxclosebrace}\\
\hllin{\ \ 50\ }}\hlstd{\ \ \ \ \ \ \ \ }\hlstd{\usebox{\hlboxopenbrace}\%\ block\ about\ \%\usebox{\hlboxclosebrace}\usebox{\hlboxopenbrace}\%\ endblock\ \%\usebox{\hlboxclosebrace}\\
\hllin{\ \ 51\ }}\hlstd{\ \ \ \ \ \ \ \ }\hlstd{\usebox{\hlboxopenbrace}\%\ block\ lecture\ \%\usebox{\hlboxclosebrace}\usebox{\hlboxopenbrace}\%\ endblock\ \%\usebox{\hlboxclosebrace}\\
\hllin{\ \ 52\ }\\
\hllin{\ \ 53\ }}\hlstd{\ \ \ \ }\hlstd{}\hlkwa{\usebox{\hlboxlessthan}/div\usebox{\hlboxgreaterthan}}\\
\hllin{\ \ 54\ }\hlstd{\\
\hllin{\ \ 55\ }}\hlstd{\ \ \ \ }\hlstd{}\hlkwa{\usebox{\hlboxlessthan}div\ }\hlstd{}\hlkwb{class}\hlstd{=}\hlstr{"footer"}\hlstd{}\hlkwa{\usebox{\hlboxgreaterthan}}\\
\hllin{\ \ 56\ }\hlstd{}\hlstd{\ \ \ \ \ \ \ \ }\hlstd{}\hlkwa{\usebox{\hlboxlessthan}p\usebox{\hlboxgreaterthan}}\hlstd{Denis\ Kostylev,\ }\hlnum{2015}\hlstd{}\hlkwa{\usebox{\hlboxlessthan}/p\usebox{\hlboxgreaterthan}}\\
\hllin{\ \ 57\ }\hlstd{}\hlstd{\ \ \ \ \ \ \ \ }\hlstd{}\hlkwa{\usebox{\hlboxlessthan}p\usebox{\hlboxgreaterthan}}\hlstd{icons\ by\ }\hlkwa{\usebox{\hlboxlessthan}a\ }\hlstd{}\hlkwb{href}\hlstd{=}\hlstr{"http://icons8.com"}\hlstd{}\hlkwa{\usebox{\hlboxgreaterthan}}\hlstd{icons8}\hlkwa{\usebox{\hlboxlessthan}/a\usebox{\hlboxgreaterthan}\usebox{\hlboxlessthan}/p\usebox{\hlboxgreaterthan}}\\
\hllin{\ \ 58\ }\hlstd{}\hlstd{\ \ \ \ \ \ \ \ }\hlstd{}\hlkwa{\usebox{\hlboxlessthan}p\usebox{\hlboxgreaterthan}}\hlstd{best\ view\ in\ google\ chrome}\hlkwa{\usebox{\hlboxlessthan}/p\usebox{\hlboxgreaterthan}}\\
\hllin{\ \ 59\ }\hlstd{}\hlstd{\ \ \ \ \ \ \ \ }\hlstd{}\hlkwa{\usebox{\hlboxlessthan}/div\usebox{\hlboxgreaterthan}\usebox{\hlboxlessthan}script\ }\hlstd{}\hlkwb{src}\hlstd{=}\hlstr{"\usebox{\hlboxopenbrace}\usebox{\hlboxopenbrace}\ url\textunderscore for('static',\ filename='\textunderscore js/main.js')\ \usebox{\hlboxclosebrace}\usebox{\hlboxclosebrace}"}\hlstd{\ }\hlkwb{type}\hlstd{=}\hlstr{"text/javascript"}\hlstd{}\hlkwa{\usebox{\hlboxgreaterthan}\usebox{\hlboxlessthan}/script\usebox{\hlboxgreaterthan}}\\
\hllin{\ \ 60\ }\hlstd{}\hlstd{\ \ \ \ }\hlstd{}\hlkwa{\usebox{\hlboxlessthan}/body\usebox{\hlboxgreaterthan}}\\
\hllin{\ \ 61\ }\hlstd{}\hlkwa{\usebox{\hlboxlessthan}/html\usebox{\hlboxgreaterthan}}\hlstd{}\\
\mbox{}
\normalfont
\normalsize

%\begin{figure}[h!]
%	\caption{base.html}
%\end{figure}

%\noindent
\ttfamily
\hlstd{\hllin{\ \ \ 1\ }\usebox{\hlboxopenbrace}\%\ extends\ 'base.html'\ \%\usebox{\hlboxclosebrace}\\
\hllin{\ \ \ 2\ }\ \usebox{\hlboxopenbrace}\%\ block\ disciplines\ \%\usebox{\hlboxclosebrace}\\
\hllin{\ \ \ 3\ }}\hlstd{\ \ \ \ \ \ \ \ }\hlstd{}\hlkwa{\usebox{\hlboxlessthan}div\ }\hlstd{}\hlkwb{class}\hlstd{=}\hlstr{"switch\textunderscore view"}\hlstd{}\hlkwa{\usebox{\hlboxgreaterthan}}\\
\hllin{\ \ \ 4\ }\hlstd{}\hlstd{\ \ \ \ \ \ \ \ \ \ \ \ }\hlstd{}\hlkwa{\usebox{\hlboxlessthan}div\ }\hlstd{}\hlkwb{class}\hlstd{=}\hlstr{"list\ not\textunderscore pressed"}\hlstd{\ }\hlkwb{onclick}\hlstd{=}\hlstr{"switch\textunderscore to\textunderscore list()"}\hlstd{}\hlkwa{\usebox{\hlboxgreaterthan}\usebox{\hlboxlessthan}span\ }\hlstd{}\hlkwb{class}\hlstd{=}\hlstr{"glyphicon\ glyphicon{-}menu{-}hamburger"}\hlstd{}\hlkwa{\usebox{\hlboxgreaterthan}\usebox{\hlboxlessthan}/span\usebox{\hlboxgreaterthan}\usebox{\hlboxlessthan}/div\usebox{\hlboxgreaterthan}}\\
\hllin{\ \ \ 5\ }\hlstd{}\hlstd{\ \ \ \ \ \ \ \ \ \ \ \ }\hlstd{}\hlkwa{\usebox{\hlboxlessthan}div\ }\hlstd{}\hlkwb{class}\hlstd{=}\hlstr{"grid"}\hlstd{\ }\hlkwb{onclick}\hlstd{=}\hlstr{"switch\textunderscore to\textunderscore grid()"}\hlstd{}\hlkwa{\usebox{\hlboxgreaterthan}\usebox{\hlboxlessthan}span\ }\hlstd{}\hlkwb{class}\hlstd{=}\hlstr{"glyphicon\ glyphicon{-}th"}\hlstd{}\hlkwa{\usebox{\hlboxgreaterthan}\usebox{\hlboxlessthan}/span\usebox{\hlboxgreaterthan}\usebox{\hlboxlessthan}/div\usebox{\hlboxgreaterthan}}\\
\hllin{\ \ \ 6\ }\hlstd{}\hlstd{\ \ \ \ \ \ \ \ }\hlstd{}\hlkwa{\usebox{\hlboxlessthan}/div\usebox{\hlboxgreaterthan}}\\
\hllin{\ \ \ 7\ }\hlstd{\\
\hllin{\ \ \ 8\ }}\hlstd{\ \ \ \ \ \ \ \ }\hlstd{\usebox{\hlboxopenbrace}\%\ for\ discipline\ in\ discipline\textunderscore list\ \%\usebox{\hlboxclosebrace}\\
\hllin{\ \ \ 9\ }}\hlstd{\ \ \ \ \ \ \ \ \ }\hlstd{\usebox{\hlboxopenbrace}\%\ set\ post\textunderscore loop\ =\ loop\ \%\usebox{\hlboxclosebrace}\\
\hllin{\ \ 10\ }}\hlstd{\ \ \ \ \ \ \ \ \ \ \ \ }\hlstd{\usebox{\hlboxopenbrace}\%\ include\ }\hlstr{"card.html"}\hlstd{\ \%\usebox{\hlboxclosebrace}\\
\hllin{\ \ 11\ }}\hlstd{\ \ \ \ \ \ \ \ }\hlstd{\usebox{\hlboxopenbrace}\%\ endfor\ \%\usebox{\hlboxclosebrace}\\
\hllin{\ \ 12\ }\\
\hllin{\ \ 13\ }\ \usebox{\hlboxopenbrace}\%\ endblock\ \%\usebox{\hlboxclosebrace}}\\
\mbox{}
\normalfont
\normalsize

%\begin{figure}[h!]
%	\includegraphics[height=0.1cm]{}
%	\caption{index.html}
%\end{figure}

%\noindent
\ttfamily
\hlstd{\hllin{\ \ \ 1\ }\usebox{\hlboxopenbrace}\%\ block\ card\ \%\usebox{\hlboxclosebrace}\\
\hllin{\ \ \ 2\ }}\hlstd{\ \ \ \ }\hlstd{}\hlkwa{\usebox{\hlboxlessthan}div\ }\hlstd{}\hlkwb{class}\hlstd{=}\hlstr{"lec\textunderscore card\textunderscore grid"}\hlstd{}\hlkwa{\usebox{\hlboxgreaterthan}}\\
\hllin{\ \ \ 3\ }\hlstd{}\hlstd{\ \ \ \ \ \ \ \ }\hlstd{}\hlkwa{\usebox{\hlboxlessthan}div\ }\hlstd{}\hlkwb{class}\hlstd{=}\hlstr{"preview"}\hlstd{\ }\hlkwb{onclick}\hlstd{=}\hlstr{"location.href='\usebox{\hlboxopenbrace}\usebox{\hlboxopenbrace}\ discipline.code\ \usebox{\hlboxclosebrace}\usebox{\hlboxclosebrace}'\ "}\hlstd{}\hlkwa{\usebox{\hlboxgreaterthan}\usebox{\hlboxlessthan}/div\usebox{\hlboxgreaterthan}}\\
\hllin{\ \ \ 4\ }\hlstd{\\
\hllin{\ \ \ 5\ }}\hlstd{\ \ \ \ \ \ \ \ }\hlstd{}\hlkwa{\usebox{\hlboxlessthan}p\ }\hlstd{}\hlkwb{onclick}\hlstd{=}\hlstr{"location.href='\usebox{\hlboxopenbrace}\usebox{\hlboxopenbrace}\ discipline.code\ \usebox{\hlboxclosebrace}\usebox{\hlboxclosebrace}'\ "}\hlstd{}\hlkwa{\usebox{\hlboxgreaterthan}}\hlstd{\usebox{\hlboxopenbrace}\usebox{\hlboxopenbrace}\ discipline.name\ \usebox{\hlboxclosebrace}\usebox{\hlboxclosebrace}}\hlkwa{\usebox{\hlboxlessthan}/p\usebox{\hlboxgreaterthan}}\\
\hllin{\ \ \ 6\ }\hlstd{}\hlstd{\ \ \ \ \ \ \ \ }\hlstd{}\hlkwa{\usebox{\hlboxlessthan}div\ }\hlstd{}\hlkwb{class}\hlstd{=}\hlstr{"control"}\hlstd{}\hlkwa{\usebox{\hlboxgreaterthan}}\\
\hllin{\ \ \ 7\ }\hlstd{}\hlstd{\ \ \ \ \ \ \ \ \ \ \ \ }\hlstd{}\hlkwa{\usebox{\hlboxlessthan}ul\usebox{\hlboxgreaterthan}}\\
\hllin{\ \ \ 8\ }\hlstd{}\hlstd{\ \ \ \ \ \ \ \ \ \ \ \ \ \ \ \ }\hlstd{}\hlkwa{\usebox{\hlboxlessthan}li\ }\hlstd{}\hlkwb{class}\hlstd{=}\hlstr{"read"}\hlstd{\ }\hlkwb{alt}\hlstd{=}\hlstr{"Read\ lecture"}\hlstd{\ }\hlkwb{onclick}\hlstd{=}\hlstr{"location.href='\usebox{\hlboxopenbrace}\usebox{\hlboxopenbrace}\ discipline.code\ \usebox{\hlboxclosebrace}\usebox{\hlboxclosebrace}'\ "}\hlstd{}\hlkwa{\usebox{\hlboxgreaterthan}\usebox{\hlboxlessthan}/li\usebox{\hlboxgreaterthan}}\\
\hllin{\ \ \ 9\ }\hlstd{}\hlstd{\ \ \ \ \ \ \ \ \ \ \ \ \ \ \ \ }\hlstd{}\hlkwa{\usebox{\hlboxlessthan}li\ }\hlstd{}\hlkwb{class}\hlstd{=}\hlstr{"download"}\hlstd{\ }\hlkwb{alt}\hlstd{=}\hlstr{"Download\ PDF"}\hlstd{\ }\hlkwb{onclick}\hlstd{=}\hlstr{"location.href='\usebox{\hlboxopenbrace}\usebox{\hlboxopenbrace}\ discipline.pdf\ \usebox{\hlboxclosebrace}\usebox{\hlboxclosebrace}'"}\hlstd{}\hlkwa{\usebox{\hlboxgreaterthan}\usebox{\hlboxlessthan}/li\usebox{\hlboxgreaterthan}}\\
\hllin{\ \ 10\ }\hlstd{}\hlstd{\ \ \ \ \ \ \ \ \ \ \ \ \ \ \ \ }\hlstd{}\hlkwa{\usebox{\hlboxlessthan}li\ }\hlstd{}\hlkwb{class}\hlstd{=}\hlstr{"share"}\hlstd{\ }\hlkwb{alt}\hlstd{=}\hlstr{"Get\ direct\ link"}\hlstd{\ }\hlkwb{onclick}\hlstd{=}\hlstr{"alert('Under\ construction')"}\hlstd{}\hlkwa{\usebox{\hlboxgreaterthan}\usebox{\hlboxlessthan}/li\usebox{\hlboxgreaterthan}}\\
\hllin{\ \ 11\ }\hlstd{}\hlstd{\ \ \ \ \ \ \ \ \ \ \ \ }\hlstd{}\hlkwa{\usebox{\hlboxlessthan}/ul\usebox{\hlboxgreaterthan}}\\
\hllin{\ \ 12\ }\hlstd{}\hlstd{\ \ \ \ \ \ \ \ }\hlstd{}\hlkwa{\usebox{\hlboxlessthan}/div\usebox{\hlboxgreaterthan}}\\
\hllin{\ \ 13\ }\hlstd{}\hlstd{\ \ \ \ }\hlstd{}\hlkwa{\usebox{\hlboxlessthan}/div\usebox{\hlboxgreaterthan}}\\
\hllin{\ \ 14\ }\hlstd{\usebox{\hlboxopenbrace}\%\ endblock\ \%\usebox{\hlboxclosebrace}}\\
\mbox{}
\normalfont
\normalsize

%\begin{figure}[h!]
%	\caption{card.html}
%\end{figure}
