\addtolength{\hoffset}{-1.7mm} % горизонтальное смещение всего текста как целого
\usepackage[utf8]{inputenc}
\usepackage[english,russian]{babel} %определение языков в документе
\usepackage{amssymb,amsmath,amsfonts,latexsym,mathtext} %расширенные наборы математических символов
\usepackage{cite}  %"умные" библиографические ссылки (сортировка и сжатие)
\usepackage{indentfirst} %делать отступ в начале параграфа
\usepackage{enumerate}  %создание и автоматическая нумерация списков
\usepackage{tabularx}  %продвинутые таблицы
\usepackage{showkeys}  %раскомментируйте, чтобы в документе были видны ссылки на литературу, рисунки и таблицы
\usepackage[labelsep=period]{caption} %заменить умолчальное разделение ':' на '.' в подписях к рисункам и таблицам
\usepackage[onehalfspacing]{setspace} %"умное" расстояние между строк - установить 1.5 интервала от нормального, эквивалентно
\renewcommand{\baselinestretch}{1.24}
\usepackage[dvips]{graphicx} %разрешить включение PostScript-графики
\graphicspath{{pics/}} %относительный путь к каталогу с рисунками, это может быть мягкая ссылка
\parindent=1.25cm

\usepackage{subcaption}
\usepackage{highlight}

\usepackage{fancyhdr}
\pagestyle{fancy}
\fancyhf{}
\fancyhead[R]{\thepage}
\fancyheadoffset{0mm}
\fancyfootoffset{0mm}
\setlength{\headheight}{17pt}
\renewcommand{\headrulewidth}{0pt}
\renewcommand{\footrulewidth}{0pt}
\fancypagestyle{plain}{ 
    \fancyhf{}
    \rhead{\thepage}}
\setcounter{page}{2} % начать нумерацию страниц с №5


\usepackage{pdflscape}
\usepackage{geometry} %способ ручной установки полей
\geometry{top=2cm} %поле сверху
\geometry{bottom=2.5cm} %поле снизу
\geometry{left=2.5cm} %поле справа
\geometry{right=2cm} %поле слева

\usepackage{booktabs}

\usepackage{hhline}
\usepackage{multirow}
\usepackage{multicol}
\setlength{\multicolsep}{-7pt}

\usepackage[usenames,dvipsnames]{xcolor}
\usepackage{afterpage}

\makeatletter
\bibliographystyle{unsrt} %Стиль библиографических ссылок БибТеХа - нумеровать в порядке упоминания в тексте Заменяем библиографию с квадратных скобок на точку в списке литературы
\renewcommand{\@biblabel}[1]{#1.}
\makeatother

%Меняем везде перечисления на цифра.цифра
\renewcommand{\theenumi}{\arabic{enumi}}
\renewcommand{\labelenumi}{\arabic{enumi}}
\renewcommand{\theenumii}{\arabic{enumii}}
\renewcommand{\labelenumii}{\arabic{enumi}.\arabic{enumii}.}
\renewcommand{\theenumiii}{\arabic{enumiii}}
\renewcommand{\labelenumiii}{\arabic{enumi}.\arabic{enumii}.\arabic{enumiii}.}

\righthyphenmin=2 % Минимальное число символов при переносе - 2.

%titles settings
\usepackage{titlesec}
 
\titleformat{\section}
    {\normalsize\bfseries}
    {\thesection} 
    {1em}
    {\filcenter}{}
 
\titleformat{\subsection}
    {\normalsize\bfseries}
    {\thesubsection}
    {1em}{}
 
% Настройка вертикальных и горизонтальных отступов
%\titlespacing*{\section}{\parindent}{*4}{*4}
%\titlespacing*{\subsection}{\parindent}{*4}{*4}

\definecolor{codebg}{gray}{0.9}
\usepackage{listings}
\lstset{
language=Delphi,
basicstyle=\small\sffamily, % размер и начертание шрифта для подсветки кода
numbers=left,               % где поставить нумерацию строк (слева\справа)
numberstyle=\footnotesize,           % размер шрифта для номеров строк
stepnumber=1,                   % размер шага между двумя номерами строк
numbersep=5pt,                % как далеко отстоят номера строк от подсвечиваемого кода
backgroundcolor=\color{codebg}, % цвет фона подсветки - используем \usepackage{color}
showspaces=false,            % показывать или нет пробелы специальными отступами
showstringspaces=false,      % показывать или нет пробелы в строках
showtabs=false,             % показывать или нет табуляцию в строках
frame=single,              % рисовать рамку вокруг кода
tabsize=2,                 % размер табуляции по умолчанию равен 2 пробелам
captionpos=t,              % позиция заголовка вверху [t] или внизу [b] 
breaklines=true,           % автоматически переносить строки (да\нет)
breakatwhitespace=false, % переносить строки только если есть пробел
escapeinside={\%*}{*)}   % если нужно добавить комментарии в коде
}
\renewcommand{\lstlistingname}{Листинг}
