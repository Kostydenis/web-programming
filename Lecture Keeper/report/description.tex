\part{Общее описание приложения}
\section{Используемые инструменты}

В разработке сайта был использован язык программироания Python и микрофреймворк Flask, который отвечал за обработку запросов клиента. Также в состав данного фреймворка входит шаблонизатор Jinja2.

Также использовался CSS-препроцессор - SASS. Он позволяет соблюдать прицип DRY (англ. Don't repeat yorself - не повторять свой код), предоставляя расширенные возможности, отсутствующие в CSS, а именно:
\begin{itemize}
	\item переменные - существует возможность один раз задать значение переменной, затем многократно использовать в коде;
	\item миксины - "функции" языка, сохранив целые части кода, можно использовать их многократно, в том числе с использованием переменных. Также существует ряд готовых библиотек (в данной работе использовалась библиотека Bourbon);
	\item вложенные стили и наследование - они позволяют использовать правила одного элемента в другом.
\end{itemize}

\section{Файловая структура приложения}
Файловая структура веб-приложения имеет следующий вид:

\begin[Bash]{lstlistings}
	Lecture\ Keeper
	├── public_html
	│   ├── _bourbon
	│   ├── _sass
	│	│	├── main.scss
	│   ├── app
	│   │   ├── __init__.py
	│   │   ├── disciplines.py
	│   │   ├── static
	│   │   │   ├── _css
	│	│	│	├── main.scss
	│   │   │   ├── _fonts/
	│   │   │   ├── _js/
	│   │   │   │   └── main.js
	│   │   │   ├── _pics/
	│   │   │   │   └── _icons
	│   │   │   └── pdfs/
	│   │   ├── templates/
	│   │   │   ├── about.html
	│   │   │   ├── base.html
	│   │   │   ├── card.html
	│   │   │   ├── electr/
	│   │   │   ├── ics/
	│   │   │   ├── index.html
	│   │   │   └── web/
	│   │   └── views.py
	│   ├── config.py
	│   └── run.py
	└── report/
		├── pics/
		│	├── screens/
		│	└── sketches/
		├───main.tex
		└───main.pdf
\end{lstlistings}

Запуск сервера осуществялется с помощью файла run.py:

\begin[Bash]{lstlistings}
	python3 run.py
\end{lstlistings}

После запуска сервера приложения, оно становится доступно по адресу: http://localhost:5000. Для того, чтобы сделать сервер доступным в сети или включить режим отладки, необходимо внести следующие строки в файл config.py соответственно:

\begin[Python]{lstlistings}
	host=0.0.0.0 #адрес компьютера в сети
	debug=True
\end{lstlistings}

Каждая страница сайта формируется при запросе пользователя, шаблоны находятся по адресу /public_html/app/templates/. Общий шаблон для всех страниц base.html отвечает за отображение панели навигации, бокового меню, контейнера с содержимым страницы и футера.

Главная страница сформирована из шаблона /public_html/app/templates/index.html, в котором содержатся шаблоны "карточек дисциплин" равным количеству дисциплин, зарегистрированных в приложении.

Информация, для отображения страницы "About" находится в файле about.html. Тексты лекций находятся в соответствующих подкаталогах каталога /templates. Также существует возможность загрузки pdf-файла каждой лекции, которые находятся в директории /static. Также в этой папке находятся файлы, отвечающие за оформление страниц (/_css/main.css) и за интерактивное поведение элементов страниц (/_js/main.js). В каталоге /fonts расположены файлы гарнитур, используемые на страницах сайта.

В каталогах /_sass и /bourbon находятся нескомпилированные файлы стилей и библиотека готовых стилей соответственно. 